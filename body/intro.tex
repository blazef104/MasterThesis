\chapter{Introduction}



\bigskip
In \hyperref[chap:1]{Chapter~\ref*{chap:1}} malware analysis techniques will be introduced. The chapter will have a particular focus on evasive malwares as they poses particular challenges in the dynamic analysis process. Common evasive and fingerprinting techniques will be presented together with some environment testing suites. 


\bigskip
In \hyperref[chap:2]{Chapter~\ref*{chap:2}} the modern virtualization methods will be presented with a particular focus on QEMU and full system emulation. An overview of the internal working mechanism of QEMU will be presented as well as of its intermediate language TCG. This translator and the QEMU structure are at the base of the Panda framework and generally of dynamic analysis frameworks based on QEMU.


\bigskip
In \hyperref[chap:3]{Chapter~\ref*{chap:3}} the panda-re analysis framework will be presented with particular focus on its plugin system and record/replay capabilities. 


\bigskip
In \hyperref[chap:4]{Chapter~\ref*{chap:4}} 


\bigskip
In \hyperref[chap:5]{Chapter~\ref*{chap:5}} 