\chapter{Introduction}

Malicious Software (malware) has become increasingly common, and profitable, in the latest years. Usually malware are differentiated based on their goal, some examples are: adware dedicated to displaying advertisement on the user pc, spyware are the ones dedicated to monitoring the user, ransomware are the ones encrypting important files and asking for a ransom and so on. 

In the past it was possible to clearly differentiate malware types based on their goal and their complexity. A modern example are Adwares, these used to be really simple pieces of code, targeted to a broad group of people, hoping to generate the maximum profit. However, with the years even such pieces of malware started to become more sophisticated and to embed complex mechanism to avoid detection, evade anti-virus software and even package more advanced features\cite{bitdef}.

In order to detect such malicious pieces of code it is important to understand how they works and interact with the system. The source code of a malware is almost never available for review therefore a different approach must be taken to understand the internal mechanism of a program when only the final executable is available. 

An executable, in order to run on a system, must be compiled. The compilation is the process of translating the source code of a program, written in a high level language, into instructions that can be executed directly by the CPU. The process of understanding how a program works by reading and interpreting CPU instructions "by hand" is called Reverse Engineering.

Malware Analysis is the process of analyzing a malicious piece of code in order to extract as many information as possible from it. This is especially valuable to understand the capabilities of a malware but also to extract useful Indicator of Compromise (IoCs) that can be leveraged by to detect an infection of a system. 

Malware analysis and detection is, therefore, a cat and mouse game in which the two parties engage in a never ending chase trying to keep up with one another. For this reason it is particularly important that the analyst's arsenal is always up to date. Malware Analysis is composed of many different phases and approaches depending on what is the final goal and the time available for the analyst. The two main phases are static analysis and dynamic analysis. Static analysis involves using different programs to extract information from the sample as well as reading through the assembly code. This is a very slow approach in which both the tools and the analyst can be easily fooled by the malware author in drawing wrong conclusions. For this reason dynamic analysis plays a major role in understanding the real capabilities of a binary. The main advantage is that, besides being faster than static analysis, IoCs can be collected in a semi-automated way by running the sample in carefully crafted environments. These are specialized virtual machines which allows the analyst to perform deep introspection of the running system.

Obviously the malware author does not want the program to be analysed and the more they are able to slow down the analysis the more are the chances that the malware can spread and achieve its goal. For this reason modern malware employ many different techniques in order to avoid detection and make the work of the analyst harder. When it comes to dynamic analysis, a particularly effective techniques consists in fingerprinting the environment in order to detect the type of hardware and software present in the system. In this way the malware can decide if the environment resembles a real one or not and hide its real capabilities accordingly.

As a matter of facts virtual machines are the core of the dynamic analysis process as they allow to run the malware on controlled and isolated environments. Complex analysis systems are built on top of Virtual Machines based on Whole System Emulation frameworks which allows to have fine grained control not only over the code that is being run but also on the hardware. This has many benefits; for example, since the hardware is emulated, it is possible to carry out malware analysis of many different samples, even the ones designed for a different architecture, on top of an arbitrary system. Moreover the ability to tap into different components and place custom pieces of code in very specific points of the emulation process, such as when the instructions are translated between one architecture and the other, allows to build very powerful and complex systems that are virtually invisible to the malware.

The most common Whole System Emulation tool is QEMU which is an open source CPU and hardware emulator. This hardware abstraction step introduces some overhead but this is nothing compared to the great level of detail that can be achieved with this type of analysis. 

However, the fact that the system is virtualized inevitably introduces some artifacts in the environment. These can be of different types depending if they are introduced at hardware or software level. As a matter of fact hardware artifacts are present because the processors and other components are virtualized. On the other hand software artifacts are generated because some additional drivers are often required on the guest system if it to run properly. 

The objective of this thesis is to enhance an existing analysis framework based on whole system emulation in order to make it completely transparent to the malware. This framework called Platform for Architecture Neutral Dynamic Analysis or PANDA is based on QEMU and leverages a plugin system to perform many different analysis techniques. My contribution consists of adding anti-evasion plugins making it possible to run even evasive malware in such framework and taking advantage of existing plugins to perform a complete analysis.


\bigskip
In \hyperref[chap:2]{Chapter~\ref*{chap:2}: \nameref{chap:2}} the most common techniques used to dissect malicious code are presented. This chapter will have a particular focus on evasive malware as they poses specific challenges in the dynamic analysis process. The most common evasive and fingerprinting techniques will be analyzed. The available environment testing suites will also be presented. 


\bigskip
In \hyperref[chap:3]{Chapter~\ref*{chap:3}: \nameref{chap:3}} the modern virtualization methods will be presented with a particular focus on QEMU and its whole system emulation capabilities. A short overview of the internal working mechanism of QEMU and its intermediate language TCG will be given. This translator and the QEMU structure are particularly important as they are at the base of the Panda framework and generally of dynamic analysis frameworks based on QEMU.


\bigskip
In \hyperref[chap:4]{Chapter~\ref*{chap:4}: \nameref{chap:4}} the panda-re analysis framework will be presented with particular focus on its plugin system and record/replay capabilities. Moreover a small comparison between this framework and other available ones will be given. 


\bigskip
In \hyperref[chap:5]{Chapter~\ref*{chap:5}: \nameref{chap:5}}  my contribution to the PANDA framework will be presented together with some of the challenges in extending its functionalities to cover the most common anti analysis techniques. In the end the results of running some anti-analysis test programs in the new version of the framework will be presented.


\bigskip
In \hyperref[chap:6]{Chapter~\ref*{chap:6}: \nameref{chap:6}} the conclusions and possible future works to further refine and enhance the framework are illustrated.